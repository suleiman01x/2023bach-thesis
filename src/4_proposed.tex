\chapter{提案手法}
\label{proposed}

本章では提案手法について述べる.

\section{概要}
本研究では遅延を減らすのではなく,遅延を前提として快適に演奏を行うことができるシステムを目指す.
その手法として演奏相手のリズム予測を行い,本来の演奏と同じタイミングで予測音を再生するシステムを提案する.

\section{遅延,ジッター対処としての予測}

\section{演奏予測の今後}

\section{Adaptive Metronome (既存手法)の課題点}

\section{仮説}
遅延のあるネットワーク下でも,演奏相手の演奏を予測しながら演奏を行うことができれば,遅延の量に関係なくまるで同じ部屋にいるかのように演奏できると仮説を立てる.

従来のAdaptive Metronomeと違い,メトロノームの音ではなく予測の音を再生することで演奏者はまるで人間の演奏相手と演奏しているかと同等の体験を得られることができると考える.
本システムを用いると音楽における音のニュアンス,表現を保ちつつ,遅延の補償を行うことができる.
なお音楽的な表現を保つことが目的であるため,表現を伝達するのに十分な予測精度を得られればよいと考える.

%%% Local Variables:
%%% mode: japanese-latex
%%% TeX-master: "../bthesis"
%%% End:
