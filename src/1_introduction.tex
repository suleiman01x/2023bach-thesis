\chapter{序論}
\label{introduction}

本章では本研究の背景,課題及び手法を提示し,本研究の概要を示す.

\section{はじめに}
\label{introduction:background}

世界的なインターネット,SNSの普及がもたらしたグローバルなコミュニケーションの発展により,世界中のアーティストをつなぐグローバルな音楽のコミュニティが形成している.
それだけでなく,2019年の新型コロナウイルス感染症の広がりで多くの人々は自宅から行う遠隔のコミュニケーションを強いられ,Apple,Metaを含む多くの企業によるインターネットで人々をリアルタイムでつなげる仮想世界への関心が上がっている.
これらの状況を踏まえて今まで以上にインターネットを介した複数の演奏者同士による遠隔にリアルタイムで行う音楽演奏のシステムへの需要が上がっている.
こうした異なる場所にいるミュージシャンをネットワークを介して,同じ部屋にいるのと同じように演奏する音楽演奏を「ネットワーク音楽演奏 (Network Music Performance)\cite{lazzaro}」と呼ぶ.

ネットワーク音楽演奏において遅延は致命的な問題である.
遅延があると演奏者同士のタイミングがずれ,リズムが崩れてしまう.
相手が演奏している音をリアルタイムで聴き,それに合わせて演奏することが重要な音楽演奏においてはこのような遅延があると演奏は不可能になる.
\cite{latency:effect}などの研究によると100ms以上になると演奏が困難になる.
この遅延の問題を解消するためにYamahaのSyncroom\cite{syncroom}やZoomのS6 SessionTrakなどの遅延を減らすシステムが開発されてきた.
しかし音声のバッファ,演奏者のネット環境,演奏者同士の距離などの様々な原因で遅延が発生してしまうことは避けられない.

その状況を踏まえて本研究ではネットワーク音楽演奏における遅延を減らすのではなく,遅延を前提として演奏に影響を与えないシステムを提案する.

\section{本論文の構成}

本論文における以降の構成は次の通りである.

~\ref{background}章では,背景を述べる.
~\ref{related}章では,本研究における問題提起を述べる.
~\ref{proposed}章では,本研究の提案手法を述べる.
~\ref{implementation}章では,~\ref{proposed}章で述べたシステムの実装について述べる.
~\ref{evaluation}章では,本システムの実験評価を行い,考察する.
~\ref{conclusion}章では,本研究のまとめと今後の課題についてまとめる.


%%% Local Variables:
%%% mode: japanese-latex
%%% TeX-master: "../thesis"
%%% End:
