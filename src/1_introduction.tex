\chapter{序論}
\label{introduction}

本章では本研究の背景,課題及び手法を提示し,本研究の概要を示す.

\section{はじめに}
\label{introduction:background}

世界的なインターネット,SNSの普及がもたらしたグローバルなコミュニケーションの発展により,世界中のアーティストをつなぐグローバルな音楽のコミュニティが形成している.
それだけでなく,2019年の新型コロナウイルス感染症の広がりで多くの人々は自宅から行う遠隔のコミュニケーションを強いられ,Apple,Metaを含む多くの企業によるインターネットで人々をリアルタイムでつなげる仮想世界への関心が上がっている.
これらの状況を踏まえて今まで以上にインターネットを介した複数の演奏者同士による遠隔にリアルタイムで行う音楽演奏のシステムへの需要が上がっている.
こうした遠隔に行う音楽演奏を「ネットワーク音楽演奏 (Network Music Performance)\cite(lazarro)」と呼ぶことにする.

ネットワーク音楽演奏においては,遅延は致命的な問題である.
遅延があると演奏者同士のタイミングがずれ,リズムが崩れてしまう.
相手が演奏している音をリアルタイムで聴き,それに合わせて演奏することが重要な音楽演奏においてはこのような遅延があると演奏は不可能になる.
研究によると80ms以上になると演奏が困難になる.
YamahaのSyncroomやZoomのS6 SessionTrakなど,このような遅延を減らすシステムは多数開発されてきたが,これらには限界がある.
パケットが速く届いてもジッターの関係で安定して情報を処理するまでにバッファは必要であるうえ,人それぞれのネット回線の混雑度,住む場所などで遅延が発生してしまう場合がある.
それ以前に光の速さより速く伝達することはできないため,最低限光が地球を回る速度以上の遅延が発生してしまう.

本研究ではネットワーク音楽演奏における遅延を減らすのではなく,遅延を前提として演奏に影響を与えないシステムを提案する.
その手法として本研究では演奏相手の演奏を予測し,その予測を再生することで遅延を補償することを提案する.
\cite{nmpbook}
\cite{admet}
\cite{admet:experiment}
\cite{alexandraki:2013}
\cite{alexandraki:2014}
\cite{tablanet}
これにより演奏者は相手の演奏の予測を聴き,それに合わせて演奏することで遅延を感じることなく演奏を行うことができる.

\section{本論文の構成}

本論文における以降の構成は次の通りである.

~\ref{background}章では,背景を述べる.
~\ref{related}章では,本研究における関連研究を述べる.
~\ref{proposed}章では,本研究の提案手法を述べる.
~\ref{implementation}章では,~\ref{proposed}章で述べたシステムの実装について述べる.
~\ref{evaluation}章では,本システムの実験評価を行い,考察する.
~\ref{conclusion}章では,本研究のまとめと今後の課題についてまとめる.


%%% Local Variables:
%%% mode: japanese-latex
%%% TeX-master: "../thesis"
%%% End:
