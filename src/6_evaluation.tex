\chapter{実験}
\label{evaluation}
本章では,4人の被験者に対して行った提案システムの実験について評価を行う.

\section{提案システムの実験}
4人の被験者に対して,提案システム,Adaptive Metronome,従来の遅延つきのネットワーク音楽演奏システムの実験,評価を行った.

\section{手法}
\subsection{概要}
被験者の中から3組のペアを作り,それぞれに対して0ms,80ms,160msの遅延で本提案システム,Adaptive Metronome,遅延つきのネットワーク音楽演奏システムの9通りでスネアドラムデュエットの演奏を行ってもらい,各演奏のあとにアンケートを取った.
各演奏履歴を元にした演奏の分析,アンケートの結果を参照して本システムの評価を行った.

\subsection{被験者}
行った実験の被験者のプロフィールは以下の通りである.
\begin{itemize}
  \item ドラム未経験者,ベース奏者 (A)
  \item ドラム経験3年程度 (B)
  \item ドラム経験3年程度 (C)
  \item ドラム経験10年程度 (D)
\end{itemize}

この4人から3組のペアを作り実験を行なった.

\begin{itemize}
  \item A―D
  \item B―C
  \item B―D
\end{itemize}

\subsection{演奏内容}
\ref{background:nmp}で記述した通り楽譜のない生の音楽セッション形式を前提としているため,その状況を可能な限り再現すると同時に,当日に覚えやすいルーティンを開発し,各組に演奏させた.

まず演奏開始前にペアはリーダー,フォロワーに分け,1小節ごとに別々の内容を演奏する.
メトロノームを用いてカウントダウンを行い,楽曲開始時に

\section{結果}
\section{評価}


%%% Local Variables:
%%% mode: japanese-latex
%%% TeX-master: "./thesis"
%%% End:
