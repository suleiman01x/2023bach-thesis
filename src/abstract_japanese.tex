卒業論文要旨 - 2023年度 (令和5年度)
\begin{center}
\begin{large}
\begin{tabular}{|M{0.97\linewidth}|}
    \hline
      \title \\
    \hline
\end{tabular}
\end{large}
\end{center}

~ \\

現在,今まで以上にネットワークを介した遠隔で複数の演奏者がリアルタイムで行う音楽演奏への注目が高まっている.
これらの演奏をネットワーク音楽演奏と呼ぶ.
音楽演奏において遅延は致命的な問題であり,時に演奏を不可能にする.
しかし,ネットワークを介したグローバルな演奏を前提とするなら,遅延は避けられない.
本研究ではネットワーク音楽演奏における遅延を減らすのではなく,遅延を前提として演奏に影響を与えないシステムを提案する.
その過程で,ネットワーク音楽演奏における遅延の問題について考察し,過去の研究を紹介する.
そしてそれらの研究をふまえたうえで,本研究では演奏予測を用いたシステムを提案する.
また実験の結果,演奏予測を用いたシステムは,遅延がある状況下でも演奏の崩れを抑えることができることを示し,本研究の有効性を示した.

~ \\
キーワード:\\
\underline{1. ネットワーク音楽演奏},
\underline{2. 遅延},
\underline{3. 演奏予測},
\underline{4. OSC}
\begin{flushright}
\dept \\
\author
\end{flushright}
