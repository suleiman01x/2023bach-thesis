卒業論文要旨 - 2023年度 (令和05年度)
\begin{center}
\begin{large}
\begin{tabular}{|M{0.97\linewidth}|}
    \hline
      \title \\
    \hline
\end{tabular}
\end{large}
\end{center}

~ \\

現在,今まで以上にネットワークを介した遠隔で複数の演奏者がリアルタイムで行う音楽演奏への注目が高まっている.
これらの演奏をネットワーク音楽演奏と呼ぶ.
ネットワーク音楽演奏においては,演奏者同士の間に遅延が発生してしまい,これは演奏に位相ずれを発生させてしまい,演奏の崩れを引き起こし,ネットワーク音楽演奏にとっては致命的な課題である.
しかし遅延というのはネットワークを介しているグローバルな演奏を前提とすると,避けることができない.
本研究ではネットワーク音楽演奏における遅延を減らすのではなく,遅延を前提として演奏に影響を与えないシステムを提案する.
その過程で,ネットワーク音楽演奏における遅延の問題について考察し,過去の研究を紹介する.
それらの研究をふまえたうえで,本研究では演奏予測を用いたシステムを提案する.
また実験の結果,演奏予測を用いたシステムは,遅延がある状況下でも演奏の崩れを抑えることができることを示し,本研究の有効性を示した.

~ \\
キーワード:\\
\underline{1. ネットワーク音楽演奏},
\underline{2. 遅延},
\underline{3. 演奏予測},
\underline{4. OSC}
\begin{flushright}
\dept \\
\author
\end{flushright}
